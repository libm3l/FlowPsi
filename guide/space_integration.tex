\subsection{Numerical Approximations to Spatial Integrals}

The numerical solution of the governing equations,
Eq.~(\ref{gov_integral}), is obtained by applying the finite
volume method.  This approach is frequently used because it can
guarantee that numerical truncation errors do not violate
conservation properties.  The numerical integration of
Eq.~(\ref{gov_integral}) begins with approximations to volume and surface
integrals.  For the volume integrals a second-order midpoint rule is
used.  For example, the numerical integration of $Q$ results in
\begin{equation}
\int_{\Omega_c(t)} Q(\Tilde{x},t) dV = Q_c(t) \V_c(t), 
\end{equation}
where $Q_c(t)$ is the value of $Q$ at the centroid of cell $c$, and
$\V_c(t)$ is defined by
\begin{equation}
\V_c(t) = \int_{\Omega_c(t)} dV.
\label{eq:volume_exact}
\end{equation}

The numerical integration of the surface integral in
Eq.~(\ref{gov_integral}) is accomplished by summing the contributions of
each of the $\mathbf{NF}$ faces of cell $c$.  Each individual contribution is
again approximated using the midpoint rule.  The flux function itself
will require additional numerical treatment, and will be discussed in
later sections.  For now, assume that the flux can be approximated by
a function, $\hat{F}$, of conservative values to the left and right of
the face.  Given this, the numerical integration of $F=F_i-F_v$
results in the following
\begin{equation}
\int_{\partial \Omega_c(t)} F dS = \sum_{f=1}^{\mathbf{NF}_c} \int_{\partial
  \Omega_{c,f}(t)} F dS \approx 
  \sum_{f=1}^{\mathbf{NF}_c} \A_{c,f}(t) \hat{F}_{f},
\end{equation}
where the area of the face, $\A_{c,f}(t)$, is defined as
\begin{equation}
\A_{c,f}(t) = \int_{\partial \Omega_{c,f}(t)} dS.
\label{eq:area_exact}
\end{equation}

At this point, Eq.~(\ref{gov_integral}) is numerically
approximated by the equation
\begin{equation}
\frac{d}{dt} \left[\V_c(t) Q_c(t)\right] + 
\sum_{f=1}^{NF_c} \A_{c,f}(t) \hat{F}_f = 0.
\end{equation}
Notice that the differential term that remains in this equation
applies to the product of volume and conservative state vector for the
cell.  However, the variable that is the objective of these
calculations is $Q_c(t)$, not $\V_c(t) Q_c(t)$.  This problem is
solved by applying the chain rule:
\begin{equation}
\frac{d}{dt} Q_c(t) \V_c(t) = Q_c(t) \frac{d}{dt} \V_c(t) + \V_c(t)
\frac{d}{dt} Q_c(t).
\end{equation}
The derivative of volume with respect to time can be converted into a
spatial integral through the use of an identity for integration over
time-dependent domains\cite{Hansen.65}:
\begin{equation}
\begin{split}
Q_c \frac{d}{dt} \V_c(t)& = Q_c \frac{d}{dt} \int_{\Omega_c(t)} dV\\
&= Q_c \int_{\partial \Omega_c(t)} \Tilde{u}_\Omega \cdot \Tilde{n}
dS \\
&\approx Q_c \sum_{f=1}^{NF_c} \A_{c,f}(t) ({\Tilde{u}_{\Omega,f}}
\cdot \Tilde{n}_{c,f}).
\end{split}
\end{equation}
This equation, known as the geometric conservation law\cite{Thomas.78}, is
necessary for correct time integration when mesh deformation is
present.  Given this, the solution method can now be described in
terms of a system of ordinary differential equations of the form
\begin{equation}
\V_c \frac{d}{dt} Q_c = R_c,
\label{eq5:timeode}
\end{equation}
where $R_c$ is given by the expression
\begin{equation}
R_c = \sum_{f=1}^{\mathbf{NF}_c} \A_{c,f} \hat{F}_{v,f} - \sum_{f=1}^{\mathbf{NF}_c} \A_{c,f}
  \hat{F}_f  -  Q_c \sum_{f=1}^{\mathbf{NF}_c} \A_{c,f} ({\Tilde{u}_{\Omega,f}}
\cdot \Tilde{n}_{c,f}).
\label{eq5:residual}
\end{equation}

Eqs.~(\ref{eq5:timeode}) and~(\ref{eq5:residual}) describe a
system of ordinary differential equations that numerically model the
time evolution of the fluid dynamics equations when simultaneously
satisfied for all cells in the mesh.  To represent this fact, the cell
subscript $c$ will be dropped.  Thus $Q_c$ represents the fluid state
for cell $c$, while $Q$ represents the fluid states of all cells in
the mesh.  For example, while Eq.~(\ref{eq5:timeode}) represents
the cell by cell differential equations, the global system of
equations is given by
\begin{equation}
\V \frac{d}{dt} Q(t) = R(Q(t),t).
\label{eq5:timeode2}
\end{equation}


