\subsection{The Dynamic Hybrid RANS-LES (DHRL) turbulence model}

One hybridized RANS-LES modeling option provided in the {\it flowPsi}
code is based on an averaging consistent framework that can couple any
arbitrary RANS turbulence model with any arbitrary RANS model.  In
this model the RANS turbulence model is solved on the time averaged
flowfield.  In this framework there is always a consistent RANS
interpretation in all regions, even those that are using the LES
model.  To derive the DHRL approach, the instantaneous velocity
($u_i$) is decomposed into Favre-averaged $\overline{u}_i$ , resolved
fluctuating ($u^{\prime\prime}$), and unresolved fluctuating
($u^\prime_i$) components, {\it i.e.}:
\begin{eqnarray}
\overline{u}_i &=& <\hat{\rho} \tilde{u}_i>/<\hat{\rho}>\\
u^{\prime\prime}_i &=& \tilde{u}_i - \overline{u}_i\\
u^\prime_i &=& u_i - \tilde{u}_i
\end{eqnarray}
Here the angle brackets denote Reynolds averaging.  For stationary flow, this is equivalent to an infinite-time average.

The assumptions used in the derivation of the DHRL method are: (1)
negligible correlation between the resolved and unresolved
fluctuations, (2) scale similarity, {\it i.e.}
$\hat{\rho}\widetilde{\tilde{u}_i\tilde{u}_j} -
\hat{\rho}\tilde{u}_i\tilde{u}_j = \alpha \tau^{SGS}_{ij}$ and
$\hat{\rho}\widetilde{u^\prime_i u^\prime_j} = \beta
\hat{\rho}\overline{u^{\prime}_i u^{\prime}_j}$, and (3) that model
constants $\alpha$ and $\beta$ are complementary.  This yields the
following blending method for RANS and LES stresses:
\begin{eqnarray}
\label{eq:dhrlblend}
\tau_{ij} &=& \alpha \tau^{SGS}_{ij} + (1-\alpha)\tau^{RANS}_{ij},\\
\alpha &=& \underbrace{-<\hat{\rho}> \overline{u^{\prime\prime}_i u^{\prime\prime}_j} \overline{S}_{ij}}_{\text{Resolved Production}}/ \Bigg(\underbrace{\tau^{RANS}_{ij} \overline S_{ij}}_{\text{RANS Production}} - \underbrace{ \overline{\tau}^{SGS}_{ij} \overline{S}_{ij}}_{\text{Inhomogeneous Production}}\Bigg)
\label{eq:dhrlswitch}
\end{eqnarray}

where $\tau^{SGS}_{ij}$ is the subgrid stress predicted by any
candidate LES model (for MILES this is identically zero), and
$\tau^{RANS}_{ij}$ is the turbulent stress predicted by any candidate
RANS model. Refer to Bhushan and Walters\cite{Bhushan.2012} and Walters
{\it et al.} \cite{Walters.2013} for further details.

The above approach is similar to dynamic LES model coefficient
evaluation, where the secondary filter is in fact the Reynolds (or
Favre) averaging operation. The numerator in Eq. (\ref{eq:dhrlswitch})
represents the sum of production of turbulent kinetic energy ($k$) due
to the resolved turbulent scales in the flow, and
$\overline{\tau}^{SGS}_{ij} \overline{S}_{ij}$ which is the mean
component of the subgrid scale turbulent kinetic energy
production. The term in the denominator,$\tau^{RANS}_{ij}
\overline{S}_{ij}$, is the production of $k$ predicted by the RANS
model. A similar expression to Eq. (\ref{eq:dhrlblend}) is adopted for
the turbulent heat flux:

\begin{equation}
q_j = \alpha q^{SGS}_j + (1-\alpha) q^{RANS}_j
\end{equation}

Equation (\ref{eq:dhrlswitch}) indicates that the model operates in a
pure LES mode only if the resolved scale production is equal to the
predicted RANS production; otherwise, the model behaves in a
transitional mode where an additional RANS stress compensates for
reduced LES content. This leads to a smooth variation of turbulent
production across the transition region. In regions with zero LES
content, i.e. numerically steady flow, the model operates in a pure
RANS mode. Note that, according to Eq. (\ref{eq:dhrlswitch}), the
value of can become negative and/or singular. In practice, the value
of is limited to lie between 0 (pure RANS) and 1 (pure LES).

One further critical aspect of the DHRL method is that the RANS model
terms are computed based solely on the Favre-averaged flowfield. For
example, for a linear eddy viscosity model the turbulent stress and
heat flux are computed as:

\begin{eqnarray}
\tau^{RANS}_{ij} &=& \frac{2}{3} \rho k \delta_{ij} - \mu_T \left( \frac{\partial \overline{u}_i}{\partial x_j} + \frac{\partial \overline{u}_j}{\partial x_i} - \frac{2}{3} \frac{\partial \overline{u}_k}{\partial x_k} \delta_{ij}\right)\\
q^{RANS}_j &=& - k_T \frac{\partial \overline{T}}{\partial x_j}
\end{eqnarray}

Similarly, all of the terms in the model transport equations for
turbulence model dependent variables (e.g. $k$ and $\omega$) are
computed in terms of $\tilde{u}_i$, including convective and
production terms. In stationary flows, for example, the velocity field
used to compute all RANS terms is obtained from a running
time-average. Other appropriate averaging methods can be adopted for
non-stationary flows. For example, as discussed below, the DHRL model
can be run using the exponential averaging option for online
statistics, in which case the averaged terms ({\it e.g.}
$\overline{u}$ ) strictly represent time-filtered, rather than
infinite averaged, quantities.

\subsubsection{Current DHRL Implementation}

For the {\tt flowpsi} implementation the DHRL model has been
implemented and verification tests have been performed for use with
the SST $k-\omega$ model for the RANS component and MILES for the LES
component.  Wall functions are not currently supported, solid
boundaries should be modeled as {\tt viscousWall} conditions and mesh
resolution should be sufficient to resolve the mean flow features for
accurate boundary layer prediction. To implement DHRL, load the
following modules:

\begin{verbatim}
loadModule: KOmegaModel
loadModule: dhrl
\end{verbatim}


\subsubsection{DHRL Inputs}

The following inputs are required in the {\tt .vars} file to properly enable DHRL:

\begin{verbatim}
turbulence_model: SST
multi_scale: DHRL
\end{verbatim}

\subsubsection{Optional inputs for DHRL include:}

\begin{verbatim}
dhrl_source_terms: on 
\end{verbatim}

{\it comment}: This input parameter can be used to disengage the
inclusion of the RANS source terms in the governing equations by
setting to ``off''. This allows the simulation to be effectively run in
a pure LES mode, while simultaneously obtaining values for the
averaged variable field statistics, as well as the RANS model variable
fields ($k$ and $\omega$) based on the time-averaged velocity field
from the LES simulation.

\begin{verbatim}
exponentialMean: off
\end{verbatim}

{\it comment}: The standard use of DHRL is for stationary turbulent
flow. To use time-filtering instead of infinite averaging, input ``on''
for {\tt exponentialMean}.

\begin{verbatim}
meanFreq: 1000
\end{verbatim}

{\it comment}: For standard DHRL ({\tt exponentialMean: off}), the value of
{\tt meanFreq} should be set to a very large number, i.e. greater than the
maximum expected timesteps to be run in the simulation, in order to
ensure that the averaging counter does not reset during the
simulation. If time filtering is used, the value of {\tt meanFreq}
determines the filter width in terms of the number of time steps

\begin{verbatim}
meanCountReset: NO DEFAULT VALUE
\end{verbatim}

{\it comment}: If this parameter is set to a non-zero integer value,
the counter used for the time averaging will be reset to that
value. Note that the values of the averaged variables themselves will
not be changed.


\subsubsection{DHRL Outputs}

The following additional variables are available as outputs when running the DHRL model:

\begin{verbatim}
alpha_dhrl
\end{verbatim}

{\it comment}: This is the scalar value of the RANS-to-LES blending parameter


\subsubsection{Recommended DHRL Start Up Procedure}

A converged DHRL solution requires infinite-time averaged quantities
within the model formulation itself. For this reason, it is desirable
to remove start up transients as quickly as possible from the flow
statistics. The following is a recommended procedure for reducing the
overall run time required for a DHRL simulation.

\begin{enumerate}
\item (Optional) Obtain a converged or nearly converged RANS
  solution. This may be useful if, for example, the initialization may
  lead to unstable behavior for a pure LES simulation.

\item Run the DHRL model using initial solution either prescribed or
  obtained from precursor RANS simulation, using the {\tt
    dhrl\_source\_terms: off} input option. This is effectively a
  MILES simulation but with collection of averaged statistics as well
  as development of the steady $k$ and $\omega$ solution field based
  on the time-averaged velocity field. The simulation need not be run
  to statistical convergence, but should be run sufficiently long that
  LES-like behavior is clearly observed and any large-scale features
  of the initial field are no longer apparent in the results.

\item Run the DHRL model using the {\tt dhrl\_source\_terms: on}
  (default) option and setting {\tt meanCountReset: \#}. A recommended
  value for {\it meanCountReset} is 100. If the simulation proves to
  be unstable, this value can be increased.

\item Continue running DHRL, {\it making sure that for any restarts
  the meanCountReset variable is not included in the .vars
  file}. Monitor averaged quantities of interest (e.g. drag) to
  determine variation of key variables as a function of simulation
  time. Determination of statistical convergence is identical to that
  for pure LES.
\end{enumerate}

\subsubsection{Optimal settings for MILES}

The preferred method for running the DHRL model is using a MILES
(Monotone Implicit LES) approach for the subgrid model whereby the
implicit dissipation provided by the upwind scheme is used as a
subgrid model. Since the upwind operators provide a natural low pass
filter, they provide dissipation characteristics that are consistent
with the requirements of a good LES filter.  However, the standard Roe
upwinding scheme can add too much dissipation and so some sort of
dissipation lowering technique will be needed for the best results.
Currently the suggested method is to utilize the skew-symmetric flux
options provided by {\it flowPsi}.  The recommended settings for this are:

\begin{verbatim}
inviscidFlux: ssf
LDS_useUpwind: 0.20
\end{verbatim}

This will use the $4^{th}$ order skew symmetric flux blended with 20\%
of the standard upwind flux.  This amount of upwinding has been found
to have good performance as a MILES LES subgrid model.  This skew
symmetric based flux will not work reliably on tetrahedral or
prismatic element meshes, however often reasonable results can be
obtained by refining the mesh with {\tt refmesh} which splits the
tetrahedral into hexahedral elements.

%\subsubsection{Low Dissipation Scheme}
%% \begin{equation}
%% u_i = \underbrace{\overline{u}_i + u^{\prime\prime}_i}_{\hat{u}_i} + u^\prime_i
%% \end{equation}

