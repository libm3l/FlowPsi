\subsection{The Wilcox (1998) $k-\omega$ Turbulence Model}
\label{wilcox_kw}

The Wilcox (1998) two equation turbulence model\cite{WilcoxBook} is an
updated version of the 1988 model\cite{Wilcox.88}.  The main
differences between the new version and the previous 1988 version
currently in the CHEM code are in the coefficients of the dissipation
terms, which now depend on local flow conditions, and in the values of
two modeling constants.  The complete turbulence model specification
is given below.\\ 

\noindent
\textbf{Eddy Viscosity:}

\begin{equation}
\mu_T = \rho k/\omega
\label{eddy_viscosity}
\end{equation}\\

\noindent
\textbf{Turbulence Kinetic Energy:}

\begin{equation}
\frac{\partial{\rho k}} {\partial{t}} + \frac{\partial{\rho U_j k}}{\partial{x_j}} =
\tau_{ij} \frac{\partial{U_i}}{\partial{x_j}} - \rho \beta^* k \omega +
\frac{\partial}{\partial{x_j}} \left[ (\mu + \sigma^* \mu_T) \frac{\partial{k}}{\partial{x_j}}] \right]
\label{tke_equation}
\end{equation}\\

\noindent
\textbf{Specific Dissipation Rate:}

\begin{equation}
\frac{\partial{\rho \omega}} {\partial{t}} + \frac{\rho U_j \partial{\omega}}{\partial{x_j}} =
\alpha \frac{\omega}{k} \tau_{ij} \frac{\partial{U_i}}{\partial{x_j}} -
\rho \beta \omega^2 +
\frac{\partial}{\partial{x_j}} \left[ (\mu + \sigma \mu_T) \frac{\partial{\omega}}{\partial{x_j}} \right]
\label{omega_equation}
\end{equation}\\

\noindent
\textbf{Closure Coefficients and Auxiliary Relations:}

\begin{eqnarray}
\nonumber
\alpha &=& \frac{13}{25}, \hspace{0.1in}
\beta = \beta_0 f_{\beta}, \hspace{0.1in}
\beta^* = \beta_{0}^* f_{\beta^*}, \hspace{0.1in}
\sigma = 0.5, \hspace{0.1in}
\sigma^* = 0.5
\label{coefficients_1}\\
\beta_0 &=& \frac{9}{125}, \hspace{0.1in}
f_{\beta} = \frac{1 + 70 \chi_{\omega}}{1 + 80 \chi_{\omega}}, \hspace{0.1in}
\chi_{\omega} \equiv \left|{\frac{\Omega_{ij} \Omega_{jk} S_{ki}}{(\beta_{0}^* \omega)^3}} \right|
\label{coefficients_2}\\
\nonumber
\beta_{0}^* &=& \frac{9}{100}, \hspace{0.1in}
f_{\beta^*} = \left\{ \begin{array} {l@{\,:\,}r}
1
& \chi_k \leq 0 \\
\frac{1 + 680 \chi_{k}^2}{1 + 400 \chi_{k}^2}
& \chi_k > 0 \\
\end{array}, \hspace{0.1in}
\chi_{k} \equiv \frac{1}{\omega^3} \frac{\partial{k}} {\partial{x_j}} \frac{\partial{\omega}} {\partial{x_j}} \right .
\label{coefficients_3}
\end{eqnarray} \\

\begin{equation}
\epsilon = \beta^* \omega k  \hspace{0.1in} \text{and} \hspace{0.1in}
l = k^{1/2}/w
\label{coefficients_4}
\end{equation}

The mean rotation and strain rate tensors are defined as:

\begin{equation}
\Omega_{ij} = \frac{1}{2} \left( \frac{\partial{U_i}} {\partial{x_j}} - \frac{\partial{U_j}} {\partial{x_i}} \right), \hspace{0.1in}
S_{ij} = \frac{1}{2} \left( \frac{\partial{U_i}} {\partial{x_j}} + \frac{\partial{U_j}} {\partial{x_i}} \right)
\end{equation}

\noindent
and the Reynolds stress tensor is:

\begin{equation}
\tau_{ij} = 2 \mu_T S_{ij} - \frac{2}{3} \rho k \delta_{ij}
\end{equation}


The implementation of the Wilcox 1998 turbulence model into CHEM is
straightforward and requires only the modification of existing source
terms and the definition of new constants.  In addition, the
computation of the coeffcients in Eqs.
\ref{coefficients_2}-\ref{coefficients_3} was added.  The new
turbulence model is invoked from the CHEM input file as
\verb!turbulence_model: Wilcox98!.  All of the other turbulence model
options, including the choices of compressibility correction and the
production function, work with this new model.


