\subsection{Time Integration}

For the time integration scheme we utilize a Crank-Nicolson inspired
scheme or three-stage backward implicit scheme

\subsubsection{Crank-Nicolson}
For the Crank-Nicolson inspired
scheme, the residual is evaluated an an interpolated half step.
The interpolated half step, denoted by the primitive variable vector
$q^{\star}$, is computed according to the relation
\begin{equation}
  q^{\star} = \left[ T^{\star}, \vec{u}^{\star}, p^{\star} \right ],
\label{half_step}
\end{equation}
where 
\begin{equation}
\begin{aligned}
  T^{\star} &= \frac{2 T^{n+1} ~ T^{n}}{T^{n+1} + T^{n}} \\
  \vec{u}^{\star} &= \frac{\sqrt{\rho^{n+1}} \vec{u}^{n+1}+\sqrt{\rho^{n}} \vec{u}^n} {\sqrt{\rho^{n+1}}+\sqrt{\rho^n}} \\
p^{\star} &= \frac{(1+\epsilon)p^{n+1} + (1-\epsilon)p^{n}}{2}\\
\label{eq:temporalavg}
\end{aligned}
\end{equation} 
This differs from the Subbareddy {\it et al.} \cite{Subbareddy.2009}
scheme in several important aspects.  First the average is defined in
terms of pressure and temperature to improve the performance
of the algorithm when there is a more complex EoS as mentioned
previously.  Second, a harmonic averaging of temperature is utilized
as this provides stability in high speed flow cases.  The energy
consistent form of velocity averaging is retained.  Note, the
$\epsilon$ pressure biasing in the pressure averaging is used to
provide the scheme with acoustic damping which is necessary when
running at high Courant-Friedrich-Lewy (CFL) numbers.  The effect of
this term will be elucidated in later discussions.  The time
integration can then be expressed as the solution to the non-linear
equation
\begin{equation}
\mathcal{L}(q^{\star})=\frac{\V}{\Delta t} (Q^{n+1} - Q^n) - R(q^{\star}) = 0 ,
\label{Lcn}
\end{equation}
where $Q = \left[ \rho, \rho\vec{u}, \rho e_T \right]$ is the
conservative variable vector where the total energy is $e_T =
e(p,T)+(\vec{u}\cdot \vec{u})/2$.  The residual, $R$, results from the
numerically integrated flux functions given by
\begin{equation}
R = \sum_{f=1}^{nf} \A_f \left( F_{v,f} - F_{i,f} \right)
\label{resid}
\end{equation}
where $\A_f$ is the area of face $f$, and $F_v$ and $F_i$ are the
viscous and inviscid fluxes respectively.

The solution is advanced in time by solving eq. (\ref{Lcn}) by a
Newton iteration method which can be described as follows
\begin{equation}
\mathcal{L}'(q^{\star, n+1,k}) (q^{\star,n+1,k+1} - q^{\star, n+1,k}) = -\mathcal{L}
(q^{\star,n+1,k}), \label{new}
\end{equation}
for $k\geq 0$, where the Newton iteration is initialized using the
previous time step value ($q^{\star,n+1,k=0} = q^{\star,n}$). To limit
the size of the jacobian of operator $\mathcal{L}$, only the first
order parts of the flux stencils are retained.  For the inviscid flux
jacobians we utilize approximate jacobians of the HLLC flux as
described by Batten {\it et al.} \cite{Batten.2006} even when the
skew-symmetric fluxes are utilized. These jacobian approximation
errors have no effect on the solution once the Newton method is
converged.


\subsubsection{Three-stage backward scheme}
The three stage backward time stepping scheme solves equation\cite{Pulliam}
\begin{equation}
  \frac{3Q^{n+1}-4Q^{n}+Q^{n-1}}{2\Delta t}- R(q^{n+1}) = 0
\label{Thee_stage}
\end{equation}
where residual $R$ is defined by equation \ref{resid}. For URANS, this scheme is more stable
then the Crank-Nicolson scheme, can utilize larger time step, less
Newton subiterations and larger value of {\tt urelax}.

\subsubsection{Local Time-Stepping Scheme}

When running in steady timestepping mode the code utilizes a local
timestep based on estimated changes in local pressure and temperature.
This estimate is base on a first order explicit step whereby the
estimated rate of change in pressure and temperature can be determined
through solving the inverse of the jacobian of the conservative
variables with respect to the primitive variables times the source
terms formed from the invisicd and viscous fluxes.  The timestep that
will limit the change of these variables no more than the
$\mu_{relax}$ factor is computed from these rates.  If this timestep
is smaller than the specified maximum timestep, then the smaller value
will be used.
