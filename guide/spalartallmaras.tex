\subsection{Spalart-Allmaras Model}
The defining equations for this model are written as follows:

\vspace{0.2cm}
{\it Kinematic Eddy Viscosity:} $\nu_t = \tilde{\nu} f_{v1}$.

{\it Eddy Viscosity Equation:}

\begin{equation}
\frac{\partial \rho \tilde{\nu}}{\partial t}+\frac{\partial{u_j \rho \tilde{\nu}}}
{\partial x_j} =
 \rho c_{b1}\Tilde{S}\tilde{\nu} 
- \rho c_{w1} f_w \left ( \frac{\tilde{\nu}}{y} \right )^2 +
\frac{\rho}{\sigma}\frac{\partial}{\partial x_k} \Bigl[ (\nu+\tilde{\nu}) \frac
{\partial\tilde{\nu}}{\partial x_k} \Bigr ] + \frac{\rho c_{b2}}{\sigma}
\frac{\partial 
\tilde{\nu}}{\partial x_k}\frac{\partial \tilde{\nu}}{\partial x_k},
\label{Spalart-Allmaras}
\end{equation}
where the last two terms on the left hand side represent turbulent
diffusion, and tensor notation is employed (repeated indexes $j,k$
denote summations). The first term on the right is the turbulence
production, while the second term denotes the destruction due to the
presence of a wall.

{\it Closure Coefficients and Auxiliary Relations:}
\begin{multline*}
\\
c_{b1}=0.1355, \quad c_{b2}=0.622, \quad c_{v1}=7.1, \quad \sigma =2/3, \\
c_{w1}=\frac{c_{b1}}{\kappa ^{2}}+\frac{(1+c_{b2})}{\sigma},  c_{w2}=0.3,
c_{w3}=2,  \kappa=0.41, \\
f_{v1}=\frac{\chi^{3}}{\chi^{3}+c_{v1}^{3}},\quad f_{v2}=1-\frac{\chi}
{1+\chi f_{v1}}, \\
f_w=g \Bigl [ \frac{1+c_{w3}^{6}}{g^6+c_{w3}^{6}}\Bigr]^{1/6}, \quad 
\chi=\frac{\tilde{\nu}}{\nu}, \quad g=r+c_{w2}(r^{6}-r), \\
r=\frac{\tilde{\nu}}{\Tilde{S}\kappa^{2}y^{2}},\quad 
\Tilde S=S+\frac{\tilde{\nu}}{\kappa^{2}y^{2}}f_{v2}, \quad
S=\sqrt{2\Omega_{ij}\Omega_{ij}}.\\
 \label{aux01}
\end{multline*}

The tensor $\Omega_{ij}=\frac{1}{2}(\partial u_i/\partial x_j-
\partial u_j/\partial x_i)$ is one half of the vorticity tensor, and
$y$ is the distance to the closest wall surface. For simplicity, no
tripping term is included.

The value ${\tilde{\nu}}$ at the wall boundary is set to zero,
and the value of $\nu_t$ in the freestream is selected as $\nu_t=10^{-3}\nu$.
 
The corresponding integral form of Eq.~(\ref{Spalart-Allmaras}) can be
included into the system of governing equations,
Eq.~(\ref{gov_integral}).  The resulting integral equations are
solved, in a decoupled form, using the same algorithmic strategies as
applied in the mean flow equations.  We do perform a simple
transformation to make the integration of the second diffusion term in
the equation more straightforward to evaluate.  This equivalent form that transforms all diffusion operators to face operations is given as:

\begin{equation}
\rho \frac{D\tilde{\nu}}{D t} =  \rho c_{b1}\Tilde{S}\tilde{\nu} 
- \rho c_{w1} f_w \left ( \frac{\tilde{\nu}}{y} \right )^2 +
\frac{\rho}{\sigma} \left \{ \nabla \cdot \left[\bigl( \nu + (1+c_{b2}) \tilde{\nu}\bigr) \nabla \tilde{\nu}\right] - c_{b2} \tilde{\nu} \nabla \cdot (\nabla \tilde{\nu})\right\}
\end{equation}

\subsection{Compressible Form of Spalart-Allmaras}

There is also an alternative compressible form of the Spalart-Allmaras model as developed by Catris and Aupoix.  In this versionof the model, the diffusiond quantity is taken to be $\sqrt{\rho} \tilde{\nu}$ giving rise to the equation:

\begin{equation}
\rho \frac{D \tilde{\nu}} {D t} = \rho c_{b1}\Tilde{S}\tilde{\nu} 
- \rho c_{w1} f_w \left ( \frac{\tilde{\nu}}{y} \right )^2 +
\frac{1}{\sigma} \left\{ \nabla \cdot (\mu \nabla \tilde{\nu}) + \nabla\cdot \left( \sqrt{\rho}\tilde{\nu} \nabla \sqrt{\rho}\tilde{\nu}\right) + c_{b2} \left(\nabla \sqrt{\rho}\tilde{\nu}\right)^2\right\}.
\end{equation}

In this version of the model, all of the coefficients remain the same as the original Spalart-Allmaras model.
