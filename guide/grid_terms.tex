\subsection{Geometric Integrations}

In the previous section, numerical integrations over surfaces and
volumes were described without explicitly defining the formulas
used to evaluate Eqs.~(\ref{eq:volume_exact}) and~(\ref{eq:area_exact}).
Here the numerical integrations that approximate the volumes and areas
of generalized cells and faces are discussed in some detail.
Several considerations have to be made regarding
these computations.  First, when the nodes of a generalized face are
not co-planar, the geometry of the face is not uniquely specified.
A unique specification of the geometry of such faces can be obtained
by subdividing the face into triangles: using a symmetric decomposition, each
edge of the face, combined with the face centroid, creates a triangle.
This strategy allows the face geometry to be independent of
data-structures used to describe generalized meshes (the geometry is
the same regardless of the ordering of the face edges).  Using this
technique, the area for the face is determined numerically using the
following summation
\begin{equation}
\Tilde{\mathcal{A}}_{c,f} = \frac{1}{2} \sum_{e=1}^{\mathbf{NE}_f} 
(\Tilde{x}_{1,e}-\Tilde{x}_{c,f})\times(\Tilde{x}_{2,e}-\Tilde{x}_{c,f}),
\label{eq:face_area}
\end{equation}
where $\Tilde{x}_{1,e}$ and $\Tilde{x}_{2,e}$ are the positions of the
two nodes of edge $e$ in counter-clockwise order, and $\Tilde{x}_{c,f}$
is the face centroid (approximated by an edge-length-weighted sum of
edge centers).  Here the computed area is a vector, which is represented
as the product of the face normal vector and the face area,
$\Tilde{\mathcal{A}} = \mathcal{A} \Tilde{n}$.

For numerical approximations of the volume integrals, it is essential to
achieve a geometric conservation of volume, avoiding
truncation errors in integrating volumes that would yield inaccurate total
volume computations.  For example, one would expect that the sum of the
individual cell volumes would be equal to the total grid volume if exact
arithmetic were used.  To accomplish this, the volume of a cell is determined
by decomposing it into tetrahedra and summing the individual
tetrahedral volumes.  The volume of a tetrahedron is given by
\begin{equation}
\mathcal{V}_{tet} = \frac{1}{3!} \left(\Tilde{a} \cdot (\Tilde{b} \times
  \Tilde{c})\right),
\label{eq:tet_volume}
\end{equation}
where $\Tilde{a}$, $\Tilde{b}$, and $\Tilde{c}$ are the three edge vectors
from a given vertex of the tetrahedron.  The volume of a generalized
cell is then computed by using the triangulation of each face and the
cell centroid: each surface triangle and the cell centroid define a
tetrahedron.  The cell volume is the sum of the tetrahedra volumes.
Thus, the volume of a cell is computed by combining
Eqs.~(\ref{eq:face_area}) and~(\ref{eq:tet_volume}), and factoring, to obtain
\begin{equation}
\mathcal{V}_c = -\frac{1}{3} \sum_{f=1}^{\mathbf{NF}_c} (\Tilde{x}_{c,c}-
\Tilde{x}_{c,f}) \cdot \Tilde{\mathcal{A}}_{c,f},
\end{equation}
where $\Tilde{x}_{c,c}$ is the cell centroid (approximated by the
area-weighted sum of face centroids).  Notice that this computation assumes
that the face normals point outward from the cell, which explains the
negative sign. Using this
approach, one can compute the correct volume of any cell, including
highly non-convex cells.




